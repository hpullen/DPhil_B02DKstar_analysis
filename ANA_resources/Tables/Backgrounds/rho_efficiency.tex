\begin{table}
    \centering
    \begin{tabular}{ccccc}
        \toprule
        & \multicolumn{2}{c}{Signal} & \multicolumn{2}{c}{$B^0 \to D\rho^0$} \\
        & Run 1 & Run 2 & Run 1 & Run 2 \\
        \midrule
        Branching fraction & \multicolumn{2}{c}{$(4.5 \pm 0.6) \times 10^{-5}$} & \multicolumn{2}{c}{$(3.210 \pm 0.021) \times 10^{-4}$} \\
        Acceptance efficiency (\%) & $16.79 \pm 0.11$ & $17.5 \pm 0.5$ & $16.26 \pm 0.11$ & $17.0 \pm 0.5$ \\
        Selection efficiency (\%) & $1.134 \pm 0.013$ & $1.30 \pm 0.04$ & $0.454 \pm 0.007$ & $0.547 \pm 0.009$ \\
        PID efficiency (\%) & $66.53 \pm 0.16$ & $77.41 \pm 0.11$ & $2.19 \pm 0.04$ & $2.360 \pm 0.025$ \\
        \midrule
        Total efficiency & $(1.267 \pm 0.017) \times 10^{-3}$ & $(1.77 \pm 0.07) \times 10^{-3}$ & $(1.612 \pm 0.027) \times 10^{-5}$ & $(2.19 \pm 0.07) \times 10^{-5}$ \\
        \bottomrule
        \end{tabular}
        \caption{Comparison of branching fractions and efficiencies between $B^0 \to DK^{*0}$ signal decays and $B^0 \to D\rho^0$ background. Efficiencies were calculated using Monte Carlo. The ratio of expected $B^0 \to D\rho^0$ events over signal events is $(1.273 \pm 0.028) \times 10^{-2}$ for Run 1 and $(1.24 \pm 0.06) \times 10^{-2}$ for Run 2.}
\label{tab:rho_efficiency}
\end{table}
